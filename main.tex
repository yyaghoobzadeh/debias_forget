\documentclass[11pt,a4paper]{article}
\usepackage[hyperref]{acl2020}
\usepackage{times}  %Required
\usepackage{helvet}  %Required
\usepackage{courier}  %Required
\usepackage{url}  %Required
\usepackage{graphicx}  %Required
\usepackage{latexsym}
\usepackage{subcaption}
\usepackage{url}
\usepackage{stmaryrd}
\usepackage{color}

\usepackage[utf8]{inputenc} % allow utf-8 input
\usepackage[T1]{fontenc}    % use 8-bit T1 fonts
\usepackage{hyperref}       % hyperlinks
\usepackage{url}            % simple URL typesetting
\usepackage{booktabs}       % professional-quality tables
\usepackage{amsfonts}       % blackboard math symbols
\usepackage{nicefrac}       % compact symbols for 1/2, etc.
\usepackage{microtype}      % microtypography
\usepackage{multirow}
\usepackage{amssymb}
% \aclfinalcopy

% Paper-specific macros
% Paper-specific macros

%% Table
\newcommand\ent{$E$\xspace}
\newcommand\neu{$N$\xspace}
\newcommand\con{$C$\xspace}
\newcommand{\nent}{$\neg E$\xspace}

\newcommand{\bt}[2]{
	\multicolumn{1}{r}{\cellcolor{cyan!#1}{#2}}
}
\newcommand{\ws}[2]{
	\multicolumn{1}{r}{\cellcolor{red!#1}{#2}}
}


\title{Up-weighting Forgettable Examples \\ for Down-weighting
Syntactic Inference Heuristics}



% \author{%
%   Yadollah Yaghoobzadeh, Remi Tachet, T.J. Hazen, Alessandro Sordoni \\
%   Microsoft Research, Montr\'eal \\
%   \small \texttt{\{yayaghoo,retachet,tj.hazen,alsordon\}@microsoft.com}
  % examples of more authors
% }

\begin{document}

\maketitle

\begin{abstract}
It is known now that there are heuristics 
models rely on when learning inference rules
from textual data. 
We propose a simple approach to increase robustness of strong natural language inference models against some frequent heuristics in the datasets.
We find a set of hard training examples, either by the target model itself or by a more simple one, and upweight them. 
To upweight, after training on the full dataset, we  fine-tune on the hard set only.
For selecting hard sets, we propose using forgettable
examples, i.e., examples that are learned and then forgotten
during training or never learned. 
This method is simple and effective while does not rely on prior knowledge of
the heuristics. 
On challenging evaluation datasets of HANS and PAWS designed to test syntactic heuristics learned by
models, we observe significant improvements in out-of-distribution generalization learning from
MNLI and QQP  datasets.
In addition, we make several novel findings about the performance in this setting, e.g.,
models with more parameters 
are more robust; longer training
results in more robustness; there is large correlation
between hardness and premise-hypothesis word overlap; it is important to include the hard examples at the beginning of the training.


\end{abstract}

\section{Introduction}


Neural network models have become ubiquitous in natural language processing applications, pushing the state-of-the-art in a large variety of tasks involving natural language understanding (NLU) and generation \cite{wu2016google,wang2019superglue}.
In the past year, significant improvements have been obtained by  training increasingly larger neural network language models on huge amounts of data openly available on the web and then fine-tuning those base models for each downstream task \cite{devlin2018bert,peters2018deep,liu2019multi}.

In spite of their impressive performance, empirical evidence suggests that these models are far from forming human-like representations of natural language. In fact, their predictions have been shown to be brittle on examples that slightly deviate from the training distribution but are still syntactically and semantically valid \cite{jia2017adversarial,linzen2019right}. In the context of natural language inference, evidence exists that they may not be robust when tested on examples obtained by applying simple meaning-preserving transformations such as passivization \cite{dasgupta2018evaluating}.
%As pointed out by~\cite{mitchell2018extrapolation}, NLP models should ``embody the symmetries that allow the same meaning to be expressed within multiple grammatical structures''.
%Another hypothesis is that current models do not learn~\emph{compositionality} rules \cite{montague1970universal},~i.e. how to compose smaller chunks into larger units of meaning, which has been linked to their poor systematic generalization capabilities \cite{loula2018rearranging,lake2017generalization,baan2019realization,hupkes2018learning}.
Increasing evidence supports the hypothesis that these models mainly tend to capture task- and dataset-specific biases such as shallow lexical word overlap features \cite{poliak2018hypothesis,dasgupta2018evaluating,linzen2019right,clark2019dont,zhang-etal-2019-paws}, which seems to be at odds with the common belief that they form high-level semantic representations of the input data \cite{bengio2009learning}. The reliance on highly predictive but brittle features is not confined to NLU tasks, it is also a perceived shortcoming of image classification models \cite{brendel2019approximating,geirhos2018imagenet,jacobsen2018excessive}. A relevant recent attempt at achieving robust learning when multiple ``views'' of the same training data are available can be found in \newcite{arjovsky2019invariant}.

Our general goal is to investigate whether it is possible to train NLU models that are more robust to distribution shifts. In particular, we investigate the possibility to identify a set of ``hard'' or ``atypical'' examples, which would unlikely be explained by simple heuristics and, if identified correctly and up-weighted during training, could enable learning more robust features. In the past, dataset re-sampling and weighting techniques have been studied in order to solve class imbalance problem~\cite{chawla2002smote} or covariate shift~\cite{sugiyama2007covariate}, notably by importance weighted empirical risk minimization. However, it has also been shown that up-weighting hard examples may be dangerous in the presence of outliers or noise \cite{chapelle2007training,Kumar10,toneva2018empirical}.

Concurrently to our work, \newcite{clark2019dont,he2019unlearn} and \newcite{mahabadi2019simple} give evidence towards the effectiveness of reweighting examples in building more robust NLU models. The authors generally assume~\emph{a priori} knowledge of the heuristics in the dataset and specifically weight examples that cannot be explained by those heuristics. In this work, we explore whether examples considered hard by ``weak'' or ``simple'' models (e.g. parametric models with a small number of parameters) naturally exclude the dataset heuristics without any prior knowledge of them. The underlying assumption is that weak models can more easily capture simple explanations of the training data but tend to underfit more complex patterns.

We consider~\emph{example forgetting} \cite{toneva2018empirical} as a model-dependent measure of ``hardness'' of an example. For a given task (\textit{e.g.} image classification), example forgetting is defined as the number of times the neural network shifts from properly classifying an example to making a mistake on the same example at the next training epoch. Examples with at least one forgetting event, the \emph{forgettable} examples, are rather atypical compared to the \emph{unforgettable} ones that contain very common features, prototypical of the class (\textit{e.g.} an occluded gray plane versus a white plane centered on a bright blue sky). It is interesting to note that forgetting events capture the dynamics of example learning, and not solely their loss at the end of training, as considered in \newcite{clark2019dont}. In this paper, we investigate whether up-weighting forgettable (aka hard) examples can help in training more robust models.  

\noindent
Our contributions are the following:
\begin{itemize}
    \item We first extend the results of \newcite{toneva2018empirical} by computing forgetting events in two popular NLP datasets,
    namely MNLI \cite{williams2017broad}, a natural language inference dataset, and QQP \cite{qqp}, a paraphrase dataset, for various architectures of increasing capacity. Our results show that weaker models tend to overfit on known heuristics for those datasets, a fact which contrasts with the observations by \newcite{toneva2018empirical} in the image setting, and open paths for future investigations.
    \item We propose a new training method to increase the robustness of NLP models, consisting in fine-tuning models on the forgettable examples of weaker models. Our approach does not assume \emph{a priori} knoweldge
    about the existing heuristics in the datasets. We apply our method to BERT and XLNET models and demonstrate a significant gain in performance on HANS and PAWS, two recent test sets designed to assess the robustness of language models: our best models achieve better performance than BERT, XLNET and recently proposed robust models.
    \item Finally, we show that the large versions of BERT and XLNET outperform their base counterparts on HANS and PAWS, demonstrating that larger models not only improve generalization but also robustness. Those large models also benefit from our method. For instance, \xlnetlarge goes from 76.1\% to 86.08\% on HANS. To the best of our knowledge, these are the first evaluations on HANS of \bertlarge and XLNET.
\end{itemize}

% We first extend the results of \newcite{toneva2018empirical} by computing forgetting events in MNLI \cite{williams2017broad}, a natural language inference dataset, and QQP \cite{qqp}, a paraphrase dataset, for various architectures of increasing capacity. The robustness of our models is verified by considering their performance on the recently proposed HANS and PAWS test sets. HANS \cite{linzen2019right} contains linguistically correct inference problems that cannot be solved by simple common heuristics, such as lexical overlap, usually learnt by models trained on the MNLI training set. Similarly, PAWS \cite{zhang-etal-2019-paws} contains paraphrase and non-paraphrase pairs with high lexical overlap, and is extremely challenging for models trained on the QQP training set. Our best models achieve better performance on the HANS and PAWS datasets than BERT, XLNET and recently proposed robust models. We also uncover interesting insights on forgettable examples for these datasets, which contrast with the observations by \newcite{toneva2018empirical} in the image setting, and open paths for future investigations.

% We first extend the results of \newcite{toneva2018empirical} by computing forgetting events in MNLI \cite{williams2017broad}, a natural language inference dataset, and QQP \cite{qqp}, a paraphrase dataset, for various architectures of increasing capacity. The robustness of our models is verified by considering their performance on the recently proposed HANS and PAWS test sets. HANS \cite{linzen2019right} contains linguistically correct inference problems that cannot be solved by simple common heuristics, such as lexical overlap, usually learnt by models trained on the MNLI training set. Similarly, PAWS \cite{zhang-etal-2019-paws} contains paraphrase and non-paraphrase pairs with high lexical overlap, and is extremely challenging for models trained on the QQP training set. Our best models achieve better performance on the HANS and PAWS datasets than BERT, XLNET and recently proposed robust models. We also uncover interesting insights on forgettable examples for these datasets, which contrast with the observations by \newcite{toneva2018empirical} in the image setting, and open paths for future investigations.

% Each such occurrence is coined a \emph{forgetting event}.
% The authors show that on various image datasets, the distribution of forgetting events is rather striking: numerous examples are never forgotten -- that is, once properly classified they remain so for the rest of training, while others withstand many a forgetting event.
% Visually inspecting those \emph{forgettable examples} shows they are rather atypical compared to the \emph{unforgettable} ones that are fairly prototypical of a class (\textit{e.g.} an occluded gray plane versus a white plane centered on a bright blue sky).
 




\section{Robustness against common syntactic heuristics}
\subsection{Problem definition}
We address natural language inference (NLI) and its special case of paraphrasing (PP).
Models train on datasets like MNLI for NLI or QQP for PP
are prune to learn some syntactic heuristics as it 
is shown in \newcite{linzen2019right,zhang-etal-2019-paws}.
The main heuristics is the word overlap between hypothesis and premis. \newcite{naik2018stress} study some misclassified examples and find that word overlap is the major reason (29\% of cases). For NLI, there is high correlation between large word overlap and entailment and very low word overlap and neutral.
Here we try to make more robust models to perform well when tested on examples violating these heuristics.

We assume that the evaluation set  has 
a good number of supporting examples as a minority in the training set.
So we evaluate on HANS as NLI and PAWS as PP datasets.
For training, we use MNLI and QQP. 



\begin{figure}[tbp]
\centering
\includegraphics[scale=0.55]{figures/acl.pdf}
\caption{Proposed training for obtaining more robust natural language inference models. We first detect~\emph{forgettable}, or~\emph{hard}, examples from the MNLI training set using a weak model. These are likely to be examples that cannot be solved by simple heuristics. Then, after having fine-tuned a target model on the MNLI training set, we perform an auxiliary round of fine-tuning only on the forgettable subset to obtain a more robust model.}
\label{fig:method}
\end{figure}

\subsection{Our method}
We start from pretrained networks like BERT or XLNet.
In Fig \ref{fig:method}, the basic elements of our method is 
shown. 
We first find a set of forgettable examples using a model, 
then upweight those when training our target model by fine-tuning 
exclusively on them after fine-tuning on all examples first.

The model used for computing forgettables could be any  
simple classifier.
We do not design that model manually and since our targeted heuristics are supposed to be very common, most models' forgettables are violating those heuristics, and when we do the second stage of fine-tuning on forgettables, the target model is adapted to more challenging cases. 
Our method does not introduce a new hyperparameters and it is 
simple compared to other related work. 

To compute forgettables, we follow the original work's algorithm \cite{toneva2018empirical}
 and track the number of times each example is forgotten, following the same procedure described in \citet{toneva2018empirical}. In short, an example is forgotten if it goes from being correctly to incorrectly classified during training (because of gradient updates performed on other examples). 

If an example is forgotten at least once or is never learnt during training, we call it ``forgettable''. 
To remove the effect of shifting label distributions, we randomly sample forgettable examples for each label to keep the label distributions uniform
% of the forgettable sets identical to the ones of MNLI or QQP 
(i.e., 33\% from each of the three labels in MNLI and 50\% for the paraphrase label in QQP). 




\section{Experimental setup}
\label{seupt}

\subsection{Inference datasets: MNLI and HANS}
\label{sec:dataset}
MNLI~\citep{williams2017broad} is a popular NLU dataset containing more than 400,000 premise/hypothesis pairs annotated with textual entailment information (neutral, entailment or contradiction). Multiple studies have hypothesized that deep learning models tend to capture simple heuristics from the MNLI training data, and do not build an actual understanding of the task. Over the years, a series of diagnostic datasets have been released to test these hypotheses.
The most recent of these datasets, HANS~\citep[Heuristic Analysis for NLI Systems]{linzen2019right}, contains curated templates designed to test the robustness of a model against the following three heuristics for recognizing if a premise entails a hypothesis: lexical overlap (a premise entails any hypothesis built from a subset of its words), subsequence (a premise entails any of its contiguous subsequences) and constituent (a premise entails all the complete subtrees in its parse tree). In particular, any model relying exclusively on those heuristics would not have a higher than chance classification accuracy on this test set.

\citet{linzen2019right} show that a variety of existing models -- including BERT~\citep{devlin2018bert}, the state-of-the-art model at the time -- perform, overall, only slightly better than chance on classification accuracy of HANS evaluation data. This confirms the hypothesis that models trained on MNLI data tend to learn the three aforementioned heuristics rather than actually understanding the task. To test our methodology, we thus make use of the HANS evaluation dataset, which contains 30,000 examples equally split between the two labels: ``entailment'' and ``non-entailment''.\footnote{In HANS, ``contradiction'' and ``neutral'' labels are combined into ``non-entailment''.}

\subsection{Paraphrase datasets: QQP and PAWS}
\label{sec:dataset_qqp}

The QQP corpus~\citep{qqp} is another widely used NLU dataset containing over 400,000 pairs of sentences annotated as either paraphrase or non-paraphrase. As a consequence of the dataset design, pairs with high lexical overlap have a high probability of being paraphrases. Similarly to MNLI, models trained on QQP are thus prone to learning lexical overlap as a highly informative feature and do not capture the common sense underlying paraphrasing. 

In order to diagnose this phenomenon, \citet{zhang-etal-2019-paws} propose PAWS (Paraphrase Adversaries from Word Scrambling), a question paraphrase dataset containing over 108,000 sentence pairs well-balanced with respect to the word overlap heuristic. Models with high performance on QQP have been shown to perform terribly on PAWS' evaluation set: for example, the accuracy of BERT is around 91.3\% on QQP and only 32.2\% on PAWS (Table \ref{tab:paws}). This makes it an interesting test-bed for our method. 

In more details, PAWS has 2 sub-tasks, PAWS-wiki and PAWS-QQP. We use PAWS-QQP and report performance on its development set, which contains 677 questions pairs. To be comparable to results from~\citet{zhang-etal-2019-paws}, we reproduce the QQP split introduced by~\citet{wang2017bilateral} and we report results both in precision-recall area under the curve (AUC) and accuracy (Acc) (see Section \ref{sec:paws}). It is also important to note that the PAWS-QQP development set is not balanced. Training examples from PAWS were never used to update our models.



\subsection{Simple baselines for finding forgettables}
% as: here you need to restress that the simple baselines will serve as "bias models", i.e. we hope they will capture biases in the dataset and forgettable example will correspond to those example that do not match the biases (contribution of the paper basically)
We train two models, bag-of-words (BoW) and bidirectional LSTM (BiLSTM), as our simple baselines to compute forgetting statistics of different examples in the training set. We use the term \textit{simple} to emphasize the fact that these models have fewer parameters than the ones we will eventually consider (Section \ref{sec:strong}). Our conjecture is that networks with lower capacity will discover the samples that strongly support the various heuristics described in Sections \ref{sec:dataset} and  \ref{sec:dataset_qqp}. In particular, \emph{forgettable} examples for those models will correspond to sentences that do not verify said heuristics.

Both models are Siamese networks, with similar input representations and classification layers.
For the input layer, we lower case and tokenize the inputs into words and initialize their representations with Glove, a 300 dimensional pretrained embedding~\citep{pennington2014glove}.
For the classification task, from the premise and hypothesis vectors $p$ and $h$, we build the concatenated vector $s = [p, h, |p - h|, p \odot h]$ and pass it to a 2-layer feedforward network. 
To compute $p$ or $h$, the BoW model max-pools the bag of word embeddings,
while the BiLSTM model max-pools the top-layer hidden states of a 2-layer bidirectional LSTM. The hidden size of the LSTMs is set to 200. Overall, BoW and BiLSTM contains 560K and 2M parameters respectively.

\subsection{Target models}
\label{sec:strong}
Recently, significant improvements in many NLP tasks have been achieved by deep transformer-based models pretrained on huge amounts of unlabeled data.
Examples of such models include BERT \cite{devlin2018bert}, XLNET \cite{yang2019xlnet}, RoBERTA \cite{roberta2019}, AlBERT \cite{lan2019albert} and SpanBERT \cite{spanBERT2019}.
In this work, we focus on BERT and XLNET, both the base and large versions, and consider them as our strong models. In particular, \bertbase being the model of choice in previous work \cite{clark2019dont,zhang-etal-2019-paws}, it will serve as our default architecture.

\subsection{Computing forgetting events}
\label{sec:forg_stat}

We train our two simple baselines and \bertbase on MNLI and QQP
.We show the statistics of example forgetting in Table~\ref{tab:forg_stats}. The results below all make use of the balanced forgettable sets.

\begin{figure}[t]
\centering
  \includegraphics[width=0.48\textwidth]{figures/forgetting_counts.pdf}
  \caption{Distribution of forgetting events in MNLI and QQP training set for the three models after five training epochs. As can be seen, a majority of examples are not forgotten during training. We make use of examples 
  with at least one forgetting event in our method for robust models.}
\label{fig:forgcount-freq}
\end{figure}

Table \ref{tab:forg_stats} shows the number of forgettable examples before and after balancing for BoW, BiLSTM and BERT on MNLI. It is worth noting that the larger the model, the fewer the forgettable examples. The performance of the models on the dev set of MNLI is also included, and appears negatively correlated to the number of forgettable examples. Figure~\ref{fig:forgcount-freq} displays the distribution of forgetting events for each model on MNLI, showing in particular the significant number of examples never learnt by our simple baselines. We also see in Figure~\ref{fig:wordoverlap-unforg} that unforgettable examples have high (resp. low) word overlap in the entailment (resp. non-entailment) class. Example forgetting seems to correlate well with that known, strong heuristic.

Table \ref{tab:forg_stats} also contains the number of forgettable examples for our three default models on QQP. Interestingly, BiLSTM has more forgettables than BoW even though it reaches a better final performance. This fact could be explained by QQP having noisy labels, which has been shown to skew the number of forgetting events~\citep{toneva2018empirical}. The distribution of forgetting events for each model can be found in Figure \ref{fig:forgcount-freq}. In what follows, we denote the sets of forgettable examples from BERT, BiLSTM and BoW as \fbert, \flstm and \fbow  respectively.

\subsection{Fine-tuning on forgettable examples}
\label{sec:fine_tune}
We first fine-tune our target models on the whole dataset for $3$ epochs  to get a reasonable prior for the task. We then perform an additional stage of fine-tuning for $3$ epochs on a subset of the training set, composed of the forgettable examples from \fbert, \flstm or \fbow . 



\section{Results}
\label{sec:eval}
\subsection{MNLI and HANS}
\begin{table*}[ht]
\footnotesize
\centering
\begin{tabular}{llccc}
\toprule
& \textbf{Train examples} & \textbf{HANS} & \textbf{MNLI} & \textbf{Avg.}  \\
\midrule
\small{1} & All & 58.3 & \textbf{84.5} & 71.4        \\
\midrule
\small{2} & \bertbase forgettables $_{\balancedbert}$   & 48.8                     & 38.9                         & 43.9\\
\small{3} & \hspace{0.1cm} Random $_{\balancedbert}$ & 51.9                   & 75.7                         & 63.8\\
\small{4} & BiLSTM forgettables $_{\balancedlstm}$ & 54.0                     & 66.8                         & 60.4 \\
\small{5} & \hspace{0.1cm} Random $_{\balancedlstm}$ & 51.1                   & 79.0                         & 65.1\\
\small{6} & BoW forgettables $_{\balancedbow}$    & 54.1                     & 68.3                         & 61.2 \\
\small{7} & \hspace{0.1cm} Random $_{\balancedbow}$ & 53.9                   & 79.6                         & 66.8\\
\midrule
&\emph{Additional stage of finetuning} \\
\small{8} & All + \bertbase forgettables   & 70.8                     & 81.8                         & 76.3  \\
\small{9} & All + BiLSTM forgettables &  {73.6}                     & 82.4             & {78.0} \\
\small{10} & \hspace{0.1cm} All + Random $_{\balancedlstm}$       & 60.9                     & 84.4                         & 72.7  \\
\small{11} & All + BoW forgettables    & \textbf{73.7}                     & 82.4             & \textbf{78.1} \\
\midrule
&\emph{Additional stage of finetuning (avg${_{\pm std}}$)} \\
\small{8} & All + \bertbase forgettables   
& $67.4_{\pm 2.47}$                     & $82.0_{\pm 0.25}$                         &   \\
\small{9} & All + BiLSTM forgettables 
&  $69.9_{\pm 2.24}$  & $82.7_{\pm 0.25}$             &  \\
\small{11} & All + BoW forgettables    & $70.4_{\pm 2.10}$                     & $82.5_{\pm 0.25}$             &  \\
\midrule
&\emph{From~\citet{clark2019dont}} & & & \\
\small{12} & All & 62.4 & 84.2 & 73.3 \\
\small{13} & All (reweight) & 69.2 & 83.5 & 76.4 \\
\small{14} & Learned Mixin & 64.0 & 84.3 & 74.2\\
\midrule
&\emph{From~\citet{mahabadi2019simple}} & & &  \\
\small{15} & Product of Experts & 66.5 & 84.0 & 75.3     \\
\midrule
&\emph{From~\citet{he2019unlearn}} & & &  \\
\small{15} & DRiFt-HYPO & 67.1 & 84.3 & 75.7     \\
\midrule
&\emph{From~\citet{linzen2019right} and \citet{Nangia_2019}} & & &  \\
\small{16} & Estimated MTurks & 76.0 & 92.0 & 84.0 \\
\bottomrule
\end{tabular}
\caption{Results of \bertbase model trained on different sources of training examples. For each line, the accuracy of the corresponding model is shown on MNLI dev and HANS and the average of the two.  
Line 1 replicates the original \bertbase result~\citep{devlin2018bert}.
% We denote with $\Delta$ the absolute improvement with respect to the standard setting of fine-tuning on all the MNLI examples (first row). 
Lines from 2 to 7 correspond to finetuning only on subsets of MNLI data. The third block of results (lines from 8 to 11) corresponds to first finetuning \bertbase on the entire MNLI data and then performing an additional finetuning stage on selected examples. We also compare performance to the recent baselines of \newcite{clark2019dont} (lines 12 to 14) and \newcite{mahabadi2019simple} (line 15). They obtain slightly higher results for their base model~\textrm{All} but our best model outperforms theirs.}
\label{tab:twoclass}
\end{table*}

\begin{table*}[]
\small
\centering
\begin{tabular}{lccccccc}
\toprule
\textbf{Model} & \multicolumn{3}{c}{\textbf{Entailment}} & & \multicolumn{3}{c}{\textbf{Contradiction}} \\
& \emph{All}    & \emph{High}  & \emph{Low} & & \emph{All}     & \emph{High}    & \emph{Low} \\
\midrule
All & \textbf{84.1}   & \textbf{90.2}   & \textbf{75.8}  & & 84.7    & 85.4    & \textbf{84.3} \\
All + BiLSTM forgettables & 77.9   & 82.6   & 71.5  & & \textbf{85.0}      & \textbf{87.0} & 84.0  \\   
\bottomrule
\end{tabular}
\caption{Fine-grained accuracy results of \bertbase on MNLI dev set before and after finetuning
on forgettables. 
We split the evaluation set into High ($>$ mean) and Low ($<$ mean) word-overlap examples,
where word-overlap is measures under Jaccard Index between hypothesis and premise. 
Finetuning hurts \ent{} results in both High and Low but improves \nent{} on High.
This is in line with the fine-grained results of HANS depicted in Figure \ref{fig:fine_eval_baselines} where all examples have High word-overlap.}
\label{fine_mnli}   
\end{table*}

\begin{figure}[t]
\includegraphics[scale=0.42]{figures/heuristic_plot.pdf}
\caption{Performance of tested models versus the baselines for ``entailment'' and ``non-entailment'' categories and for each heuristic present in the HANS test set. Our approach seems to better retain performance of the entailment category while still increasing accuracy over the baselines of the non-entailment category, albeit to a smaller extent.}
\label{fig:fine_eval_baselines}
\end{figure}


\iffalse
\begin{table}[ht]
\small
\caption{Results of \bertlarge model trained on different sources of training examples.}
\small
\label{tab:bertlarge}
\centering
\begin{tabular}{lccc}
\toprule
\textbf{Train examples} & \textbf{HANS} & \textbf{MNLI} & \textbf{Avg.}  \\
\midrule
All & 72.3 & 86.4 &  79.3 \\
\midrule
\emph{Additional stage of finetuning} & & &\\
All + BoW forgettables & 77.3 & 85.5 & 81.4 \\
All + BiLSTM forgettables & \underline{77.5} & \underline{85.5} & \underline{81.5} \\
\bottomrule
\end{tabular}
\end{table}
\fi

\begin{table*}[ht]
\small
\centering
\begin{tabular}{lcccccc}
\toprule
\textbf{Model} & \multicolumn{2}{c}{\textbf{HANS}} & \multicolumn{2}{c}{\textbf{MNLI}} & \multicolumn{2}{c}{\textbf{Avg.}}  \\
& \emph{base} & \emph{large} & \emph{base} & \emph{large} & \emph{base} & \emph{large} \\
\midrule
BERT & 58.3 & 72.3 & 84.5 & 86.4 & 71.4 & 79.3 \\
BERT + BoW forgettables & 73.7 & 77.3 & 82.4 & 85.5 & 78.1 & 81.4 \\
BERT + BiLSTM forgettables & 73.6 & 77.5 & 82.5 & 85.5 & 78.3 & 81.5 \\
\midrule
XLNET & 63.9$_{\pm 3.0}$ & $76.1_{\pm 2.7}$ & $86.5_{\pm 0.29}$ & 89.7$_{\pm 0.1}$ & & \\
XLNET + BoW forgettables & 73.0 & 85.0 & 84.3 & 88.2 & \\
XLNET + BiLSTM forgettables & 75.1 & 86.8 & 85.5 & 87.8 & &  \\
%\midrule
%Roberta & 46.3 & 50.2 & 87.9 & 89.9 &  & 84.15 \\
%Roberta + BoW forgettables & & & & & \\
%Roberta + BiLSTM forgettables & & & & & & \\
\bottomrule
\end{tabular}
\caption{Results of the \emph{base} and \emph{large} version of various recent NLU models trained on different sources of training examples.}
\label{tab:bertlarge}
\end{table*}



Our main results are presented in Table~\ref{tab:twoclass}.

\paragraph{Forgetting Statistics} Table \ref{tab:forg_stats} shows the number of forgettable examples before and after balancing for BoW, BiLSTM and BERT on MNLI. It is worth noting that the larger the model, the fewer the forgettable examples. The performance of the models on the development set of MNLI is also included, and appears negatively correlated to the number of forgettable examples. Figure~\ref{fig:forgcount-freq} displays the distribution of forgetting events for each model on MNLI, showing in particular the significant number of examples never learnt by our weak baselines.

\paragraph{Training on Forgettables} Lines 2 to 5 report results of fine-tuning BERT on different subsets of the MNLI dataset. This setting aligns with the setting presented in~\citet{toneva2018empirical} where the authors show that, in multiple image classification tasks, the same generalization performance can be obtained by training a model initialized randomly on its own forgettable examples. Our results suggest that this behavior may be task and/or architecture dependent: in our setting, training only forgettable examples particularly affects generalization performance on MNLI. The most extreme drop in performance is observed when BERT is only fine-tuned on its own forgettable examples (line 2) achieving an accuracy of 38.9\%. Training on BiLSTM (line 4) or BoW forgettables (line 6) examples causes a lesser drop in accuracy on MNLI although still noticeable. One of the possible reasons of the dramatic performance loss observed in line 2 is that BERT forgettables are significantly fewer than the counterparts from weaker baselines. 
In order to rule out this hypothesis, we train on a random subset of examples of the same size (\balancedbert, line 3). These results suggest that there is an intrinsic difficulty in BERT forgettables that deserves to be investigated in the future.
To some extent, this is also the case for BiLSTM and BoW forgettables (lines 4 and 6), when comparing to random samples with the same size (lines 5 and 7).

\paragraph{Additional Fine-Tuning} Lines 8-11 report the results obtained by fine-tuning a pretrained model on the set of forgettables, as described in Section \ref{sec:fine_tune}. The results confirm that slightly biasing the model towards hard examples improves robustness at a slight (albeit noticeable) drop in MNLI accuracy. 
Our best model is obtained by using the BiLSTM forgettable examples (line 9) achieving an accuracy of 73.6\% on HANS (max over 3 seeds, mean 73.7\% $\pm$ 0.5\%)
which constitutes a +15.7\% absolute improvement with respect to the base model in line 1 and +4.8\% and +7.5\% with respect to the concurrent models of \newcite{clark2019dont} and \newcite{mahabadi2019simple}. 
Results on line 10 confirm that the forgettables examples identified by BiLSTM are responsible for the improvement: fine-tuning on the same number of randomly chosen examples leads to a much smaller increase. 
Fine-tuning on BoW forgettables (line 11) is also comparable to BiLSTM forgettables (line 9).
% This shows that the choice of the weak model does not seem to matter and therefore prior knowledge about the dataset biases is not required.
An additional observation is that BERT forgettables provide less improvement in robustness than BiLSTM or BoW. 
We hypothesize that this is due to the smaller size of the BERT forgettables compared to BiLSTM or BoW.
% This seems to align with our hypothesis that weaker baselines capture simpler explanations in the dataset.

\textcolor{red}{NEED TO COMMENT ON TABLE 3. We also show the detailed results of HANS based on its three different heuristics for our best performing model (line 9) in Figure \ref{fig:fine_eval_baselines}. 
To give an example of how the various models perform, we retrieved the nearest neighbors of a given HANS example (see Appendix, Table \ref{tab:NNs}).}



\paragraph{Robustness of XLNET and larger models}
\textcolor{red}{\citet{yang2019xlnet} recently introduced XLNET, a transformer-based language model outperforming BERT in standard benchmarks. We applied our evaluation procedure to XLNET and report the results in Table \ref{tab:bertlarge}. \xlnetbase reaches 63.9\% on HANS. After fine-tuning on BiLSTM forgettables, we see an increase of +11.2\%; our method seems to transfer to other architectures.}

\textcolor{red}{Perhaps more interesting, is the effect of using even bigger models. A growing body of literature suggests that increasing the capacity of deep networks results in better generalization \cite{belkin2018reconciling,deepdouble}. These results usually assume no distribution shift between train and test sets. We investigate here whether robustness to the distribution shift studied in this paper may appear ``for free'' in models with a larger number of parameters. To that end, we apply our method to the ``large'' version of BERT and XLNET, \bertlarge and \xlnetlarge, which 
achieve very strong performance on the MNLI dataset~\cite{devlin2018bert,yang2019xlnet}. Results are also shown in Table~\ref{tab:bertlarge}. We see that the large versions generalize on HANS significantly better than their base counterparts (e.g. 76.1\% vs 63.9\% for XLNET), confirming -- in this setting -- that larger models seem more robust. We also observe a significant improvement in performance as a result of finetuning on BiLSTM forgettables, supporting the applicability of the method to larger architectures. In particular, \xlnetlarge fine-tuned on BiLSTM forgettables shows a +10.7\% increase in performance, reaching an impressive 86.8\%!}

\iffalse
\begin{table*}[t]
    \centering
\setlength{\tabcolsep}{4pt}
\begin{tabular}{lcrrrrrrr}
\toprule
\multirow{2}{*}{Training examples} 
&
& \multicolumn{2}{c}{lexical} & \multicolumn{2}{c}{subseq} & \multicolumn{2}{c}{const} \\
     \cmidrule(lr){3-4} \cmidrule(lr){5-6} \cmidrule(lr){7-8}
     & overall 
            & \ent            & \nent           & \ent            & \nent           & \ent             & \nent \\
\midrule
All  & 58.3     
& \bt{0.0}{96.3} & \bt{0.0}{38.4} 
& \bt{0.0}{99.6} & \bt{0.0}{4.7} 
& \bt{0.0}{99.7} & \bt{0.0}{10.6} \\
All + finetuning on BiLSTM forgettables & 73.6   
& \bt{0.0}{76.9} & \bt{0.0}{81.6} 
& \bt{0.0}{90.6} & \bt{0.0}{40.8} 
& \bt{0.0}{93.3} & \bt{0.0}{60.8} \\
\midrule
\newcite{mahabadi2019simple} & 66.5
& \bt{0.0}{93.5} & \bt{0.0}{61.7} 
& \bt{0.0}{96.3} & \bt{0.0}{19.2} 
& \bt{0.0}{98.4} & \bt{0.0}{30.2} \\
\newcite{clark2019dont}  & 69.2
& \bt{0.0}{67.9} & \bt{0.0}{77.4} 
& \bt{0.0}{84.3} & \bt{0.0}{44.9} 
& \bt{0.0}{81.0} & \bt{0.0}{59.6} \\
\bottomrule
\end{tabular}
\caption{Accuracy of the entailment (\ent) and non-entailment (\nent) classes on HANS for three heuristics:
    lexical overlap (lexical), subsequence overlap (subseq), and constituent overlap (const).
}
\label{tab:hans-detailed}
\end{table*}
\fi

\subsection{QQP and PAWS}
\label{sec:paws}
% QQP (Quora question paraphrase) is a paraphrase detection dataset from Quora questions. 
% Each question pair is labeled by either positive or negative. 
% Performance of strong NLP models is very high on the evaluation sets from this data. 
% Recently, PAWS is introduced \cite{zhang-etal-2019-paws} which is made using back translation and applying heuristics for cleanup. The word overlap in PAWS examples is significantly higher than QQP making it a challenging evaluation set for QQP models.
% For example, the accuracy of BERT trained on QQP is drops from around 91\% to 32\%.

In this section, we report the results of our method applied to QPP (see Section \ref{sec:dataset_qqp}).

\paragraph{Forgetting Statistics} Table \ref{tab:forg_stats} contains the number of forgettable examples for our three defaults models. Interestingly, BiLSTM has more forgettables than BoW even though it reaches a better final performance. The distribution of forgetting events for each model can be found in the appendix, Figure \ref{fig:forgcount-freq-qqp}.

\paragraph{Fine-Tuning on Forgettables} We assess the robustness of our fine-tuned models on PAWS. Results can be found in Table \ref{tab:paws}. \textcolor{red}{TO COMPLETE WITH FULL NUMBERS}. 

\begin{table*}[ht]
\small
\centering
\begin{tabular}{lcccccc}
\toprule
\textbf{Model}                           & \multicolumn{2}{c}{\textbf{QQP $\rightarrow$ QQP}} & \multicolumn{2}{c}{\textbf{QQP $\rightarrow$ PAWS}} &           \multicolumn{2}{c}{\textbf{Avg.}}         \\
                                & Acc           & AUC          & Acc           & AUC          & Acc & AUC \\
%\midrule
%Random (p=0.3624)               & 50    & 58.96 & 56.60 & 41.77 & 53.30 & 50.365 \\
%Random (p=0.5)                  & 49.79 & 62.23 & 50.52 & 46.67 & 50.16 & 54.45 \\
%Random (p=0.6376)               & 50    & 66.04 & 43.40 & 50.68 & 46.70 & 58.36 \\
\midrule
% FROM PAWS PAPER, TABLE 7:
\emph{From~\citet{zhang-etal-2019-paws}} \\
BOW                                          & 83.2 & 89.5 & 29.0 & 27.1 & 56.1 & 58.3 \\
BiLSTM                                       & 86.3 & 91.6 & 34.8 & 37.9 & 60.6 & 64.8 \\
ESIM~\citep{chen-etal-2017-enhanced}         & 85.3 & 92.8 & 38.9 & 26.9 & 62.1 & 59.9 \\
DecAtt~\citep{parikh-etal-2016-decomposable} & 87.8 & 93.9 & 33.3 & 26.3 & 60.6 & 60.1 \\
DIIN~\citep{gong2017natural}                 & 89.2 & 95.2 & 32.8 & 32.4 & 61.0 & 63.8 \\
BERT~\citep{devlin2018bert}                  & 90.5 & 96.3 & 33.5 & 35.1 & 62.0 & 65.7 \\
\midrule
\bertbase                       & 90.6 & 96.6 & 32.2 & 33.8 & 61.4 & 65.2 \\
\bertbase + BOW forgettables    & 88.8 & 95.4 & 44.8 & 36.4 & 66.8 & 65.9 \\
\bertbase + BiLSTM forgettables & 88.2 & 95.2 & 41.9 & 37.2 & 65.1 & 66.2 \\
\bertbase + BERT forgettables   & 87.7 & 95.2 & 48.4 & 34.4 & 68.1 & 64.8 \\
\midrule
\bertlarge                       & 91.1 & 96.9 & 34.6 & 35.1 & 62.9 & 66.0 \\
\bertlarge + BiLSTM forgettables & 87.3 & 94.1 & 55.2 & 35.8 & 71.3 & 65.0 \\
\midrule
\xlnetbase                       & 90.9 & 96.6 & 36.6 & 37.5 & 63.8 & 67.1 \\
\xlnetbase + BOW forgettables    & 88.5 & 95.1 & 52.9 & 39.3 & 70.1 & 67.2 \\
\xlnetbase + BiLSTM forgettables & 87.9 & 95.0 & 49.5 & 40.5 & 68.7 & 67.8 \\
\xlnetbase + BERT forgettables   & 88.6 & 95.2 & 60.4 & 38.1 & 74.5 & 66.6 \\
\midrule
\xlnetlarge                       & {91.6} & 97.0   & 49.6 & 43.4 & 70.6 & 70.2 \\
\xlnetlarge + BOW forgettables    & 87.2 & 93 & 68.9 & 50.1 & 78.1 & 71.6\\
\xlnetlarge + BiLSTM forgettables & 85.6 & 92.6 & 64.7 & 52.5 & 75.2 & 72.6\\
\bottomrule
\end{tabular}
\caption{Accuracy (\%) and AUC score (\%) of precision-recall curves are reported for QQP test set and PAWS\textsubscript{QQP} development set. {QQP $\rightarrow$ PAWS} indicates training on QQP and evaluation on PAWS. Results of the \emph{base} and \emph{large} versions of various recent NLU models trained on different sources of training examples. \soroush{I put the 1st seed in this table for base mdoels.} }
\label{tab:paws}
\end{table*}

\begin{table*}[]
\small
\centering
\begin{tabular}{lccc|ccc}
                                & \multicolumn{3}{c|}{\pph} & \multicolumn{3}{c}{\npph} \\

                                & All    & High   & Low   & All     & High    & Low     \\
\toprule

All                       &    &    &   &     &     &     \\
All + BiLSTM forgettables &    &    &   &     &     &     \\   
\bottomrule
\end{tabular}
\caption{Fine-grained accuracy results of \bertbase on QQP dev set before and after finetuning on forgettables. 
We split the evaluation set into High ($>$ mean) and Low ($<$ mean) word-overlap examples,
where word-overlap is measures under Jaccard Index between two sentences. 
%COPIED FROM TABLE 3 fine_mnli: Finetuning hurts \pph{} results in both High and Low but improves \npph{} on High.
%This is in line with the fine-grained results of HANS depicted in Figure \ref{fig:fine_eval_baselines} where all examples have High word-overlap.
}
\label{fine_qqp}   
\end{table*}


\subsection{FEVER}
\label{sec:fever}

In this section, we report the results of our method applied to FEVER.

\begin{table*}[ht]
\small
\centering
\begin{tabular}{llcccc}
\toprule
& \textbf{Model} & \multicolumn{1}{c}{\textbf{FEVER-Dev}} & \multicolumn{1}{c}{\textbf{Symm-v1}} &          
\multicolumn{1}{c}{\textbf{Symm-v2}} \\
\midrule
\small{1} & \bertbase & 85.6$_{\pm 0.6}$ & 55.9$_{\pm 1.0}$ & 62.0$_{\pm 1.1}$ \\
\small{2} &\bertbase + \fbow & 85.5$_{\pm 0.4}$ & 59.2$_{\pm 1.1}$ & 64.4$_{\pm 1.1}$\\
\small{3} &\bertbase + \flstm                 \\
\midrule
 & From \citet{utama2020mind} \\
\small{4} & \bertbase & 85.8$_{\pm 0.1}$ & 57.9$_{\pm 1.1}$ & 64.4$_{\pm 0.6}$\\
\small{5} & Reweighting$_{\mbox{\tiny{bigrams}}}$  & 85.5$_{\pm 0.3}$ & 61.7$_{\pm 1.1}$ & 66.5$_{\pm 1.3}$\\
\small{6} & Regularized-conf$_{\mbox{\tiny{claim}}}$ &  86.4$_{\pm 0.2}$ & 60.5$_{\pm 0.4}$& 66.2$_{\pm 0.6}$\\
\bottomrule
\end{tabular}
\caption{}
\label{tab:fever}
\end{table*}

\section{Analysis}
\subsection{Properties of the forgettables}
Based on the results, we have:

\begin{itemize}
    \item A subset of training examples for a specific model 
     in a specific training setting
     \item The model either learns and then forget them during training at least once, or never leans them.
     \item If we start from a (particaially) randomly initialized model and train it only on the forgettables, the model does not generalize; in fact it performs worse than chance on the test examples.
    \item If we start from a pretrained model for the task, and finetune it on
    the forgettables, the model does better on HANS and worse on original test set.
    \item It is better if we pick the set of forgettables from a model with weaker architecture.
    \item Only around 50\% of BERT forgettables are forgettable by BiLSTM -- again showing that forgettables are model dependent. 
    
\end{itemize}

Some other hypothesis:
\begin{itemize}
    \item X is in F if there is a training example with different label but similar input feature. This is in fact
    similarly true for HANS examples: there are non-entailment examples with high word overlap, while there are many 
    examples in training set of MNLI with close input feature but entailment as label.
    \item the easy rules in U do not generalize to F or F is a set of outliers -- model does not learn a rule from them. 
    This is hypothesized by the poor results of models trained only on their Fs. However, eliminating these outliers hurt the model
    performance. 
    
\end{itemize}

\iffalse
\subsection{Nearest neighbors}
We take the embedding of CLS in BERT as the representation of an example. 
In BERT models, this embedding is fed to a linear classifier and optimized by cross entropy loss. 
Here, we find and look into the k-NNs of training examples for understanding what model is actually learning. 
\fi


\subsection{Remaining questions}
\begin{enumerate}
    \item How do our experiments align with the main findings of \newcite{toneva2018empirical} about
    forgettable and unforgettable examples in some image classifcation datasets:
    \begin{quote}
        a) there exist a large number of unforgettable examples, i.e.,
examples that are never forgotten once learnt, those examples are stable across seeds and strongly
correlated from one neural architecture to another; b) examples with noisy labels are among the most
forgotten examples, along with images with “uncommon” features, visually complicated to classify;
c) training a neural network on a dataset where a very large fraction of the least forgotten examples
have been removed still results in extremely competitive performance on the test set
    \end{quote}
\end{enumerate}

\section{Related Works}


% \paragraph{Out-of-distribution generalization}
% Recently, generalization to distributions other than the training one has become increasingly important, notably for real-world applications of NLP, vision, and ML in general. A growing body of literature focuses on the subject and shows that such generalization is far from being attained, even in seemingly simple cases. For instance, evidence exists that NLP models may not be robust when tested on examples obtained by applying basic meaning-preserving transformations such as passivization \cite{dasgupta2018evaluating}. As pointed out by~\citet{mitchell2018extrapolation}, NLP models should ``embody the symmetries that allow the same meaning to be expressed within multiple grammatical structures''. However, it is not the case, in particular, they do not learn~\emph{compositionality} rules \cite{montague1970universal},~i.e. how to compose smaller chunks into larger units of meaning, which has been linked to their poor systematic generalization capabilities \cite{loula2018rearranging,lake2017generalization,baan2019realization,hupkes2018learning}.
% While this might seem at odds with the common belief that high-level semantic representations of the input data are formed \cite{bengio2009learning}, the reliance on highly predictive but brittle features is not confined to NLU tasks. It is also a perceived shortcoming of image classification models \cite{brendel2019approximating,geirhos2018imagenet,jacobsen2018excessive}.

\paragraph{Out-of-distribution generalization}
% Recently, generalization to distributions other than the training one has become increasingly important, notably for real-world applications of NLP, vision, and ML in general.
A growing body of literature recently focused on out-of-domain generalization showing that such generalization is far from being attained, even in seemingly simple cases~\citep{geirhos2018imagenet,jia2017adversarial,dasgupta2018evaluating}. In particular, and in contrast with what ~\citet{mitchell2018extrapolation} recommend, NLP models do not seem to ``embody the symmetries that allow the same meaning to be expressed within multiple grammatical structures''. Supervised models seem to exhibit poor systematic generalization capabilities \cite{loula2018rearranging,lake2017generalization,baan2019realization,hupkes2018learning} thus seemingly lacking~\emph{compositional} behavior~\citep{montague1970universal}. While this might seem at odds with the common belief that high-level semantic representations of the input data are formed \cite{bengio2009learning}, the reliance on highly predictive but brittle features is not confined to NLU tasks. It is also a perceived shortcoming of image classification models \cite{geirhos2018imagenet,jacobsen2018excessive,brendel2019approximating}.
% Standard machine learning datasets are usually built by splitting a set of examples coming from the same distribution into a training and an evaluation set~\citep{mnist}.
In order to test systematically if machine learning models generalize beyond their training distribution, a number of datasets have been introduced in NLP and other ML applications~\cite{journals/cmig/Kalpathy-CramerHDABM15,peng2018moment,clark2019dont}.
Those test sets can be made automatically from designed grammars \cite{linzen2019right} and/or by human annotators~\cite{zhang-etal-2019-paws}. % Model performance on these out-of-distribution test sets is usually far below that on the training distribution. 

% They evaluate different aspects of generalization, e.g., covariate shift, label imbalance, etc.

\paragraph{Dataset re-weighting} Dataset re-sampling and weighting techniques have been studied in order to solve class imbalance problem~\cite{chawla2002smote} or covariate shift~\cite{sugiyama2007covariate}, notably by importance weighted empirical risk minimization. However, evidence exists that up-weighting hard examples may be dangerous in the presence of outliers or noise \cite{chapelle2007training,Kumar10,toneva2018empirical}.
In an NLP setting, \citet{clark2019dont,mahabadi2019simple} give evidence of the effectiveness of re-weighting training examples for increasing robustness. They generally assume~\emph{a priori} knowledge of the heuristics present in the dataset, and up/down-weight examples with respect to those heuristics. A relevant recent attempt at achieving robust learning when multiple ``views'' of the same training data are available can be found in \newcite{arjovsky2019invariant}.
Dataset sampling is also a trademark of curriculum learning, where training proceeds along a curriculum of samples with increasing difficulty~\citep{bengio2009curriculum}. \citet{Kumar10,lee2011learning,schaul2015prioritized,Zhao2015,Fan2017,conf/icml/KatharopoulosF18,screenerNet,jiang18mentor} have shown the concept can be quite successful in a variety of areas.

\paragraph{Unsupervised domain adaptation}
Unsupervised domain adaptation and generalization are examples works focused on transferring knowledge from a labeled source domain to an unlabeled target domain. Theoretical results~\citep{ben2010theory,mansour2012robust,conf/icml/0002CZG19} and algorithms~\citep{glorot2011domain,becker2013non,adel2017unsupervised,pei2018multi} under those settings are abundant. In NLP, \citet{jia2017adversarial} is of particular relevance.
``Our work is more related to distributionally robust optimization 
(Duchi and Namkoong, 2018; Hu et al., 2018) and the more recent group-DRO which does not assume access to target data and optimizes the worst-case performance under unknown, bounded distribution shift.''


AFLITE~\citep{bras2020adversarial} is an algorithmic method for bias removal in datasets without relying on prior knowledge about dataset. It filters out examples with high average predictability score, relating them to points with spurious biases. The resulting subset of 
examples is used as unbiased training data and shown to improve o.o.d performance across several benchmarks including HANS compared to training on full dataset. 
However, it harms the performance on in-distribution significantly and
acts as adversary to that. 
Our approach is designed to increase robustness to o.o.d while still keeping
in-distribution performance high. We do not filter out easy examples,
but down-weight them.
Other bias removal methods introduced for NLI are all based on strong 
domain and dataset assumptions. The bias-only models are 
designed specifically for each of the o.o.d or diagnostic tasks.
This means we need to know in advance about the evaluation distribution,
which is not ideally the case commonly.
In addition, all of these methods including AFLITE add new hyperparamters
like example weights or threshold of predictability 
that they need to tune in order to get their good peroramance.
Our method doesn't rely on any prior knowledge, and doesn't introduce any
new hyperparamter. 
\newcite{min2020syntactic} use syntactic tranformation of some training examples to augment training set and show it is 
effective to improve HANS.


\newcite{} suggest the term \emph{shortcut learning} to cover different 
terms in this domain. 
There are common features in the data that models learn to do the task which are not intended by humans designing the task.
Learning of these features is a shortcut to perform the task, which
doesn't generalizes to o.o.d.
Another relatod work is algorithmic \textbf{fairness}
which is related to learning biased solutions. 
We also observe in NLI  the existence of biases 
towards head groups of examples and debiasing hurts the head group results. 




% This has been studied within different areas of ML. 
% Domain adaptation techniques are trying to build more robust models against covariate shift.
% Unlabeled data from target domains might be used in these scenarios but there are also work 
% where there is no assumption about the target domain and the goal is to build a more general model which can work better for any unseen target domain. 
% Re-weighting training examples based on some criteria is a common approach to gain more generalization.
% This of course depends on our prior criteria for generalization in the task.
% In this work, we up-weight some training examples without depending on any priors about the task. 
% We up-weight examples where training model is uncertain about them during training, i.e., forgets them as training goes on.

% \paragraph{Sample Re-Weighting and Curriculum Learning}
% Recently, \newcite{clark2019dont,mahabadi2019simple} give evidence towards the effectiveness of re-weighting training examples for increasing robustness but they  generally assume~\emph{a priori} knowledge of the heuristics present in the dataset. 
% In the curriculum learning framework, training proceeds along a curriculum of training samples with increasing difficulty~\citep{bengio2009curriculum}. Since then, \citep{Kumar10,lee2011learning,schaul2015prioritized} have shown the concept can be quite successful in a variety of areas. For example, \citet{Kumar10} built a curriculum by labelling as easy the examples with a small loss. \citet{Zhao2015,conf/icml/KatharopoulosF18} relate the sample importance to the gradient norm of its loss with respect to the parameters of the network. \citet{Fan2017,screenerNet,jiang18mentor} learn a curriculum directly from data in order to minimize the task loss. \citet{jiang18mentor} also study the robustness of their method in the context of noisy examples. This relates to a rich literature on outlier detection and removal of examples with noisy labels~\citep{john1995robust,brodley1999identifying,sukhbaatar2014training,jiang18mentor}. We will provide evidence that noisy examples rank higher in terms of number of forgetting events. \cite{conf/icml/KohL17} borrow influence functions from robust statistics to evaluate the impact of the training examples on a model's predictions.

% \paragraph{Deep Generalization} The study of the generalization properties of deep neural networks when trained by stochastic gradient descent has been the focus of several recent publications~\citep{zhang2016understanding, keskar2016large, Chaudhari2016,Advani2017HighdimensionalDO}. These studies suggest that the generalization error does not depend solely on the complexity of the hypothesis space. For instance, it has been demonstrated that over-parameterized models with many more parameters than training points can still achieve low test error~\citep{huang2017densely,Wang2018} while being complex enough to fit a dataset with completely random labels~\citep{zhang2016understanding}.
% A possible explanation for this phenomenon is a form of implicit regularization performed by stochastic gradient descent: deep neural networks trained with SGD have been recently shown to converge to the maximum margin solution in the linearly separable case~\citep{Soudry2017,separable2}. In our work, we provide empirical evidence that generalization can be maintained when removing a substantial portion of the training examples and without restricting the complexity of the hypothesis class. This goes along the support vector interpretation provided by \citet{Soudry2017}.

% \paragraph{Continual Learning} \alex{If Maryia get good results, dump some stuff on continual learning, we'd need also to change the intro to make it a bit more concrete!}.

% \begin{itemize}
%    \item https://arxiv.org/pdf/1801.00904.pdf @\textbf{Remi}, here they apply the stuff they do to reinforcement learning in place of the PER (prioritized experience replay). I think it will be worth thinking about it.
%    \begin{itemize}
%        \item 2016: https://arxiv.org/pdf/1606.04232.pdf hard-removal of examples in the training set in the context of noisy labels. Model obtains low accuracy on MNIST, even lower than removing random subsets of the training set. Their model obtains $~93\%$ test accuracy using $3000$ training samples, whereas we have found that retaining $3000$ random examples for training results in a mean accuracy of $98.02\%$ across $3$ seeds with standard deviation of $0.07$. In comparison, our approach of retaining the most forgotten $3000$ examples results in $98.9\%$ accuracy.
%        \item 1999 https://arxiv.org/pdf/1106.0219.pdf identification of noisy labels , nice discussion about the difference between "noise" (to filter) and "exceptions" (to retain)
% \end{itemize}
% \end{itemize}


\iffalse
\subsection{Comparing our methodology to related work}
\begin{itemize}
    \item The recent work to tackle the brittleness of MNLI trained models all are built based on the prior knowledge about the biases
    in the training dataset. 
    
    
    Notably, \newcite{clark2019dont} introduce a 2 stage process as:
\begin{quote}
        (1)
train a naive model that makes predictions exclusively based on dataset biases, and (2) train
a robust model as part of an ensemble with the
naive one in order to encourage it to focus on
other patterns in the data that are more likely to
generalize
    \end{quote}
    
    \newcite{he2019unlearn} employs a similar approach and uses a bag-of-word, a hypotheis-only and a manually-built bias models, and fits 
    their target model to the residual of these weak baselines.
    
    \newcite{mahabadi2019simple} takes a very similar approach as in \newcite{clark2019dont}.
    
    In compared to these methods, we do not rely on known biases and prior knowledge about the dataset.
    We do however rely on weaker models to compute a more diverse set of forgettables.

\end{itemize}
\fi

\section{Discussion and Conclusion}

In this paper, we introduced a novel approach based on example forgetting to build more robust models for a natural language inference task. We  finetuned a pre-trained model on a set of ``hard'' examples selected by measuring ``example forgetting''~\citep{toneva2018empirical}. We evaluated the robustness of our approach by training exclusively using the MNLI dataset and the evaluating the model on the out-of-distribution test set of HANS~\citep{linzen2019right}. We improve \bertbase and \bertlarge performance on the challenging HANS test set by more than 15\% and 5\%, respectively. Although this paper focused on natural language inference, the method is widely applicable in other tasks and contexts. This constitutes one possible direction for future work. Moreover, we plan to analyze the forgettable examples more thoroughly to understand their special properties. Finally, we will study smoother re-weighting of the training examples and re-interpret the studied approach more formally in the context of importance weighted empirical risk minimization~\citep{sugiyama2007covariate}.

% \item investigate differences between hard examples and most forgotten examples 
%     \item smoothly integrate the finetuning phase by softly reweighting examples
%     \item compute more explicit measures of correlation between forgettables and the biases
%     \item understand why BERT forgettables suck.
%     \item can we even learn to avoid biases if no example contain counter examples of those biases ? put this in discussion
% \end{itemize}


\bibliography{bibliography}
\bibliographystyle{acl_natbib}

\appendix
\label{sec:appendix}
% \newcommand\ent{$E$}
% \newcommand\neu{$N$}
% \newcommand\con{$C$}
% \newcommand{\nent}{$\neg E$}
% \newcommand{\bt}[2]{#2}

\begin{table*}[t]
\begin{tabular}{p{\textwidth}}
\toprule
\textbf{Source HANS example} \vspace{1mm} \\
\hspace{2mm} \emph{The banker thanked the tourist.} $\longarrownot\longrightarrow$ \emph{The tourist thanked the banker.} \\
\\
\textbf{Nearest neighbors by BERT} \vspace{1mm} \\
\hspace{2mm} \emph{The model simulates the size of the pool of exchangeable base cations in the soil.} \\ \hspace{2mm} $\longrightarrow$ \emph{The model simulates the size of the pool of base cations in the dirt.} \vspace{2mm} \\
\hspace{2mm} \emph{(Or click to read my summary of Wolfe's and Rose's positions.)} \\
\hspace{2mm} $\longrightarrow$ \emph{To read my summary of Wolfe's position, click.} \\
\\
\textbf{Nearest neighbors by our Robust BERT} \vspace{1mm} \\
\hspace{2mm} \emph{He sat patiently as she talked}. $\longarrownot\longrightarrow$
\emph{She sat patiently as he spoke} \vspace{2mm} \\
\hspace{2mm} \emph{And Gates and Appiah would have to be thanked for opening the door.} \\ \hspace{2mm} $\longarrownot\longrightarrow$
\emph{Gates and Appiah were thanked for opening the door.} \\
\bottomrule
\end{tabular}
\caption{Two nearest neighbors for one HANS non-entailment ($\longarrownot\longrightarrow$)  example show-casing how our robust model (line 5 in Table 2) pushes supporting $\longarrownot\longrightarrow$ training data closer compared to standard BERT (MNLI). 
To compute nearest neighbors from the BERT models, the embedding  of the special token (CLS) is assumed as the representation of an example and cosine is used as the similarity metric.}
\label{tab:NNs}
\end{table*}

\iffalse
\begin{table*}[]
\begin{tabular}{p{0.45\textwidth} p{0.45\textwidth} }
\\
\multicolumn{2}{l}{\emph{The HANS example}}
\\
\toprule
\multicolumn{2}{l}{
\begin{tabular}[c]{@{}l@{}}
\emph{The banker thanked the tourist.} $\longarrownot\longrightarrow$ \emph{The tourist thanked the banker.} \\
\end{tabular}
}
\\
\toprule
\multicolumn{1}{c}{\textbf{Nearest neighbors of BERT (MNLI)}}
& \multicolumn{1}{c}{\textbf{Nearest neighbors of our Robust BERT}}     \\
\midrule
\begin{tabular}[t]{@{}p{0.45\textwidth}@{}}
\emph{The model simulates the size of the pool of exchangeable base cations in the soil.} $\longrightarrow$ \emph{The model simulates the size of the pool of base cations in the dirt.} \\
% \\ \emph{Label}: \ent
\end{tabular} & 
\begin{tabular}[t]{@{}p{0.45\textwidth}@{}}
\emph{He sat patiently as she talked}. $\longarrownot\longrightarrow$
\emph{She sat patiently as he spoke}
\end{tabular}
\
\\
% \cmidrule(lr){2-2}
\midrule
\begin{tabular}[c]{@{}p{0.45\textwidth}@{}}
P:	(Or click to read my summary of Wolfe's and Rose's positions.)
H:	To read my summary of Wolfe's position, click.
label: \ent
\end{tabular}
&
\begin{tabular}[c]{@{}p{0.45\textwidth}@{}}
\emph{And Gates and Appiah would have to be thanked for opening the door.} $\longarrownot\longrightarrow$
\emph{Gates and Appiah were thanked for opening the door.} \\
\end{tabular}
\\
\end{tabular}
\caption{Two nearest neighbors for one HANS non-entailment (\nent{})  example show-casing how our robust model (line 5 in Table 2) pushes supporting \nent{} training data closer compared to standard BERT (MNLI). 
To compute nearest neighbors from the BERT models, the embedding 
of the special token (CLS) is assumed as the representation of
an example and cosine is used as the similarity metric.}
\label{tab:NNs}
\end{table*}
\fi

%\begin{figure}[t]
%\centering
%  \includegraphics[width=0.45\textwidth]{figures/ex_by_forg_qqp.png}
%  \caption{Distribution of forgetting events for the three models after five training epochs on QQP. }
%\label{fig:forgcount-freq-qqp}
%\end{figure}
\begin{figure}[t]
\includegraphics[scale=0.42]{figures/heuristic_plot.pdf}
\caption{Performance of our best model, BERT fine-tuned on the BiLSTM forgettables \flstm, versus the baselines for ``entailment'' and ``non-entailment'' categories for each heuristic the HANS test set was designed to capture. Our approach seems to better retain performance for entailment while still increasing accuracy over the baselines for non-entailment, albeit to a smaller extent.}
\label{fig:fine_eval_baselines}
\end{figure}

\subsection{A closer look into models predictions on HANS}
Here we look into the confidence of entailment in HANS
and MNLI evaluation sets.

First, we show that BERT trained on all MNLI examples is able
to discriminate HANS entailments from non-entailments but with
a much higher threshold than 0.5. 
Finetuning on forgettables calibrates classification
threshold on HANS and makes 0.5 as the optimum value of threshold.

\begin{figure}
    \centering
    \includegraphics[width=0.4\textwidth]{figures/threshold_hans.pdf}
    \caption{Classification threshold vs HANS accuracy}
    \label{fig:threshold_hans}
\end{figure}

Second, we show the histogram of the models' entailment confidence 
in HANS examples.

\begin{figure}
    \centering
    \includegraphics[width=0.5\textwidth]{figures/confidence_hans_hist.pdf}
    \caption{Confidence Histogram on HANS.}
    \label{fig:confidence_hans}
\end{figure}

\end{document}

%TODO:
%- https://dawn.cs.stanford.edu/2019/03/22/glue/ (Signal 4: Dataset Slicing) 

% https://microsoft-my.sharepoint.com/:x:/r/personal/yayaghoo_microsoft_com/_layouts/15/Doc.aspx?sourcedoc=%7B41975454-abff-488b-8f55-c550ad840d9f%7D&action=edit&activeCell=%27compare_baseModels%27!D9&wdInitialSession=c219a303-a524-465a-bdd6-28ac21d1312c&wdRldC=1

% \begin{table}[h]
%     \centering
%     \begin{tabular}{l | l l}
%         Train source & HANS & MNLI \\
%         \hline
%         All & 60.0 & \\
%         BERT forgettables & & \\
%         BiLSTM forgettables                       & 54.0 & \\
%         BiLSTM forgettables + 20\% unforgettables & 57.5 & \\
%         BiLSTM forgettables + 30\% unforgettables & 57.5 & \\
%         BiLSTM forgettables + 50\% unforgettables & 57.5 & \\
%         BiLSTM forgettables + 80\% unforgettables & 57.5 & \\
%         \hline 
%         pretrian all, fine-tune on BERT forgettables & 67.8 & \\
%         pretrian all, fine-tune on BiLSTM forgettables & 69.2 & \\
%         pretrian all, fine-tune on BiLSTM 20\% unforgettables & 61.7 & \\
        
%     \end{tabular}
%     \caption{Caption}
%     \label{tab:my_label}
% \end{table}

