\documentclass[11pt,a4paper]{article}
\usepackage[hyperref]{acl2019}
\usepackage{times}  %Required
\usepackage{helvet}  %Required
\usepackage{courier}  %Required
\usepackage{url}  %Required
\usepackage{graphicx}  %Required
\usepackage{latexsym}
\usepackage{subcaption}
\usepackage{url}
\usepackage{stmaryrd}


\usepackage[utf8]{inputenc} % allow utf-8 input
\usepackage[T1]{fontenc}    % use 8-bit T1 fonts
\usepackage{hyperref}       % hyperlinks
\usepackage{url}            % simple URL typesetting
\usepackage{booktabs}       % professional-quality tables
\usepackage{amsfonts}       % blackboard math symbols
\usepackage{nicefrac}       % compact symbols for 1/2, etc.
\usepackage{microtype}      % microtypography
\usepackage{multirow}

\newcounter{notecounter}
\newcommand{\enotesoff}{\long\gdef\enote##1##2{}}
\newcommand{\enoteson}{\long\gdef\enote##1##2{{
\stepcounter{notecounter}
{\large\bf
\hspace{1cm}\arabic{notecounter} $<<<$ ##1: ##2
$>>>$\hspace{1cm}}}}}
\enoteson
\aclfinalcopy
\newcommand\alex[1]{\textcolor{red}{Aless: #1}}
\newcommand\remi[1]{\textcolor{green}{Remi: #1}}
\newcommand\yado[1]{\textcolor{orange}{Yado: #1}}

% \input macros

\def\balancedbert{17,748}
\def\balancedlstm{46,740}
\def\balancedbow{63,390}
\def\bertbase{BERT$_{\scriptsize \textrm{BASE}}$ }
\def\bertlarge{BERT$_{\scriptsize \textrm{LARGE}}$ }

\title{Robust Natural Language Inference Models \\ with Example Forgetting}

% The \author macro works with any number of authors. There are two commands
% used to separate the names and addresses of multiple authors: \And and \AND.
%
% Using \And between authors leaves it to LaTeX to determine where to break the
% lines. Using \AND forces a line break at that point. So, if LaTeX puts 3 of 4
% authors names on the first line, and the last on the second line, try using
% \AND instead of \And before the third author name.

\author{%
  Yadollah Yaghoobzadeh, Remi Tachet, T.J. Hazen, Alessandro Sordoni \\
  Microsoft Research, Montr\'eal \\
  \small \texttt{\{yayaghoo,retachet,tj.hazen,alsordon\}@microsoft.com}
  % examples of more authors
  % \And
  % Coauthor \\
  % Affiliation \\
  % Address \\
  % \texttt{email} \\
  % \AND
  % Coauthor \\
  % Affiliation \\
  % Address \\
  % \texttt{email} \\
  % \And
  % Coauthor \\
  % Affiliation \\
  % Address \\
  % \texttt{email} \\
  % \And
  % Coauthor \\
  % Affiliation \\
  % Address \\
  % \texttt{email} \\
}

\begin{document}

\maketitle

\begin{abstract}
We investigate whether example forgetting, a recently introduced measure of hardness of examples, can be used to select training examples in order to increase robustness of natural language understanding models in a natural language inference task (MNLI). We analyze forgetting events for natural language inference task. We provide evidence that examples that are most forgotten under simpler NLU models can be used to increase robustness of the recently proposed BERT model, measured by testing a MNLI trained model on HANS, a curated test set that exhibits a shift in distribution compared to the MNLI test set. Moreover, we show that, the ``large'' version of BERT is more robust than its ``base'' version but its robustness can still be improved with our approach.
\end{abstract}

\section{Introduction}

Neural network models have become ubiquitous in natural language processing applications, pushing the state-of-the-art in a large variety of tasks involving natural language understanding (NLU) and generation \cite{wu2016google,wang2019superglue}.
In the past year, significant improvements have been obtained by  training increasingly larger neural network language models on huge amounts of data openly available on the web and then fine-tuning those base models for each downstream task \cite{devlin2018bert,peters2018deep,liu2019multi}.

In spite of their impressive performance, empirical evidence suggests that these models are far from forming human-like representations of natural language. In fact, their predictions have been shown to be brittle on examples that slightly deviate from the training distribution but are still syntactically and semantically valid \cite{jia2017adversarial,linzen2019right}. In the context of natural language inference, evidence exists that they may not be robust when tested on examples obtained by applying simple meaning-preserving transformations such as passivization \cite{dasgupta2018evaluating}.
%As pointed out by~\cite{mitchell2018extrapolation}, NLP models should ``embody the symmetries that allow the same meaning to be expressed within multiple grammatical structures''.
%Another hypothesis is that current models do not learn~\emph{compositionality} rules \cite{montague1970universal},~i.e. how to compose smaller chunks into larger units of meaning, which has been linked to their poor systematic generalization capabilities \cite{loula2018rearranging,lake2017generalization,baan2019realization,hupkes2018learning}.
Increasing evidence supports the hypothesis that these models mainly tend to capture task- and dataset-specific biases such as shallow lexical word overlap features \cite{poliak2018hypothesis,dasgupta2018evaluating,linzen2019right,clark2019dont,zhang-etal-2019-paws}, which seems to be at odds with the common belief that they form high-level semantic representations of the input data \cite{bengio2009learning}. The reliance on highly predictive but brittle features is not confined to NLU tasks, it is also a perceived shortcoming of image classification models \cite{brendel2019approximating,geirhos2018imagenet,jacobsen2018excessive}. A relevant recent attempt at achieving robust learning when multiple ``views'' of the same training data are available can be found in \newcite{arjovsky2019invariant}.

Our general goal is to investigate whether it is possible to train NLU models that are more robust to distribution shifts. In particular, we investigate the possibility to identify a set of ``hard'' or ``atypical'' examples, which would unlikely be explained by simple heuristics and, if identified correctly and up-weighted during training, could enable learning more robust features. In the past, dataset re-sampling and weighting techniques have been studied in order to solve class imbalance problem~\cite{chawla2002smote} or covariate shift~\cite{sugiyama2007covariate}, notably by importance weighted empirical risk minimization. However, it has also been shown that up-weighting hard examples may be dangerous in the presence of outliers or noise \cite{chapelle2007training,Kumar10,toneva2018empirical}.

Concurrently to our work, \newcite{clark2019dont,he2019unlearn} and \newcite{mahabadi2019simple} give evidence towards the effectiveness of reweighting examples in building more robust NLU models. The authors generally assume~\emph{a priori} knowledge of the heuristics in the dataset and specifically weight examples that cannot be explained by those heuristics. In this work, we explore whether examples considered hard by ``weak'' or ``simple'' models (e.g. parametric models with a small number of parameters) naturally exclude the dataset heuristics without any prior knowledge of them. The underlying assumption is that weak models can more easily capture simple explanations of the training data but tend to underfit more complex patterns.

We consider~\emph{example forgetting} \cite{toneva2018empirical} as a model-dependent measure of ``hardness'' of an example. For a given task (\textit{e.g.} image classification), example forgetting is defined as the number of times the neural network shifts from properly classifying an example to making a mistake on the same example at the next training epoch. Examples with at least one forgetting event, the \emph{forgettable} examples, are rather atypical compared to the \emph{unforgettable} ones that contain very common features, prototypical of the class (\textit{e.g.} an occluded gray plane versus a white plane centered on a bright blue sky). It is interesting to note that forgetting events capture the dynamics of example learning, and not solely their loss at the end of training, as considered in \newcite{clark2019dont}. In this paper, we investigate whether up-weighting forgettable (aka hard) examples can help in training more robust models.  

\noindent
Our contributions are the following:
\begin{itemize}
    \item We first extend the results of \newcite{toneva2018empirical} by computing forgetting events in two popular NLP datasets,
    namely MNLI \cite{williams2017broad}, a natural language inference dataset, and QQP \cite{qqp}, a paraphrase dataset, for various architectures of increasing capacity. Our results show that weaker models tend to overfit on known heuristics for those datasets, a fact which contrasts with the observations by \newcite{toneva2018empirical} in the image setting, and open paths for future investigations.
    \item We propose a new training method to increase the robustness of NLP models, consisting in fine-tuning models on the forgettable examples of weaker models. Our approach does not assume \emph{a priori} knoweldge
    about the existing heuristics in the datasets. We apply our method to BERT and XLNET models and demonstrate a significant gain in performance on HANS and PAWS, two recent test sets designed to assess the robustness of language models: our best models achieve better performance than BERT, XLNET and recently proposed robust models.
    \item Finally, we show that the large versions of BERT and XLNET outperform their base counterparts on HANS and PAWS, demonstrating that larger models not only improve generalization but also robustness. Those large models also benefit from our method. For instance, \xlnetlarge goes from 76.1\% to 86.08\% on HANS. To the best of our knowledge, these are the first evaluations on HANS of \bertlarge and XLNET.
\end{itemize}

% We first extend the results of \newcite{toneva2018empirical} by computing forgetting events in MNLI \cite{williams2017broad}, a natural language inference dataset, and QQP \cite{qqp}, a paraphrase dataset, for various architectures of increasing capacity. The robustness of our models is verified by considering their performance on the recently proposed HANS and PAWS test sets. HANS \cite{linzen2019right} contains linguistically correct inference problems that cannot be solved by simple common heuristics, such as lexical overlap, usually learnt by models trained on the MNLI training set. Similarly, PAWS \cite{zhang-etal-2019-paws} contains paraphrase and non-paraphrase pairs with high lexical overlap, and is extremely challenging for models trained on the QQP training set. Our best models achieve better performance on the HANS and PAWS datasets than BERT, XLNET and recently proposed robust models. We also uncover interesting insights on forgettable examples for these datasets, which contrast with the observations by \newcite{toneva2018empirical} in the image setting, and open paths for future investigations.

% We first extend the results of \newcite{toneva2018empirical} by computing forgetting events in MNLI \cite{williams2017broad}, a natural language inference dataset, and QQP \cite{qqp}, a paraphrase dataset, for various architectures of increasing capacity. The robustness of our models is verified by considering their performance on the recently proposed HANS and PAWS test sets. HANS \cite{linzen2019right} contains linguistically correct inference problems that cannot be solved by simple common heuristics, such as lexical overlap, usually learnt by models trained on the MNLI training set. Similarly, PAWS \cite{zhang-etal-2019-paws} contains paraphrase and non-paraphrase pairs with high lexical overlap, and is extremely challenging for models trained on the QQP training set. Our best models achieve better performance on the HANS and PAWS datasets than BERT, XLNET and recently proposed robust models. We also uncover interesting insights on forgettable examples for these datasets, which contrast with the observations by \newcite{toneva2018empirical} in the image setting, and open paths for future investigations.

% Each such occurrence is coined a \emph{forgetting event}.
% The authors show that on various image datasets, the distribution of forgetting events is rather striking: numerous examples are never forgotten -- that is, once properly classified they remain so for the rest of training, while others withstand many a forgetting event.
% Visually inspecting those \emph{forgettable examples} shows they are rather atypical compared to the \emph{unforgettable} ones that are fairly prototypical of a class (\textit{e.g.} an occluded gray plane versus a white plane centered on a bright blue sky).
 




\section{Methodology}
\subsection{Problem definition}
We address natural language inference (NLI) and its special case of paraphrasing (PP).
Models train on datasets like MNLI for NLI or QQP for PP
are prune to learn some syntactic heuristics as it 
is shown in \newcite{linzen2019right,zhang-etal-2019-paws}.
The main heuristics is the word overlap between hypothesis and premis. \newcite{naik2018stress} study some misclassified examples and find that word overlap is the major reason (29\% of cases). For NLI, there is high correlation between large word overlap and entailment and very low word overlap and neutral.
Here we try to make more robust models to perform well when tested on examples violating these heuristics.

We assume that the evaluation set  has 
a good number of supporting examples as a minority in the training set.
So we evaluate on HANS as NLI and PAWS as PP datasets.
For training, we use MNLI and QQP. 



\begin{figure}[tbp]
\centering
\includegraphics[scale=0.55]{figures/acl.pdf}
\caption{Proposed training for obtaining more robust natural language inference models. We first detect~\emph{forgettable}, or~\emph{hard}, examples from the MNLI training set using a weak model. These are likely to be examples that cannot be solved by simple heuristics. Then, after having fine-tuned a target model on the MNLI training set, we perform an auxiliary round of fine-tuning only on the forgettable subset to obtain a more robust model.}
\label{fig:method}
\end{figure}

\subsection{Our method}
We start from pretrained networks like BERT or XLNet.
In Fig \ref{fig:method}, the basic elements of our method is 
shown. 
We first find a set of forgettable examples using a model, 
then upweight those when training our target model by fine-tuning 
exclusively on them after fine-tuning on all examples first.

The model used for computing forgettables could be any  
simple classifier.
We do not design that model manually and since our targeted heuristics are supposed to be very common, most models' forgettables are violating those heuristics, and when we do the second stage of fine-tuning on forgettables, the target model is adapted to more challenging cases. 
Our method does not introduce a new hyperparameters and it is 
simple compared to other related work. 

To compute forgettables, we follow the original work's algorithm \cite{toneva2018empirical}
 and track the number of times each example is forgotten, following the same procedure described in \citet{toneva2018empirical}. In short, an example is forgotten if it goes from being correctly to incorrectly classified during training (because of gradient updates performed on other examples). 

If an example is forgotten at least once or is never learnt during training, we call it ``forgettable''. 
To remove the effect of shifting label distributions, we randomly sample forgettable examples for each label to keep the label distributions uniform
% of the forgettable sets identical to the ones of MNLI or QQP 
(i.e., 33\% from each of the three labels in MNLI and 50\% for the paraphrase label in QQP). 




\section{Evaluation}
\label{sec:eval}

\subsection{Main result}

\begin{table*}[ht]
\caption{Results of BERT model trained on different sources of training examples. 
% We denote with $\Delta$ the absolute improvement with respect to the standard setting of fine-tuning on all the MNLI examples (first row). 
Lines from 2 to 7 correspond to finetuning only on subsets of MNLI data. The third block of results (lines from 8 to 11) corresponds to first finetuning BERT on all the MNLI data and then performing an additional stage of finetuning on selected examples. We also compare the performance to the recent baselines of~\cite{clark2019dont} (lines 12 to 14) and \cite{mahabadi2019simple} (line 15). They obtain slightly higher results for their base model~\textrm{All} but our best model outperforms their best result.}
\small
\label{tab:twoclass}
\centering
\begin{tabular}{llcccc}
\toprule
& Train examples & HANS & MNLI & HANS+MNLI  \\
\midrule
\small{1} & All & 58.3 & 84.5 & 64.8        \\
\midrule
\small{2} & BERT forgettables $_{\balancedbert}$   & 48.8                     & 38.9                         & 46.4\\
\small{3} & \hspace{0.1cm} Random $_{\balancedbert}$ & 51.9                   & 75.7                         & 57.8\\
\small{4} & BiLSTM forgettables $_{\balancedlstm}$ & 54.0                     & 66.8                         & 57.2 \\
\small{5} & \hspace{0.1cm} Random $_{\balancedlstm}$ & 51.1                   & 79.0                         & 58.0\\
\small{6} & BoW forgettables $_{\balancedbow}$    & 54.1                     & 68.3                         & 57.6 \\
\small{7} & \hspace{0.1cm} Random $_{\balancedbow}$ & 53.9                   & 79.6                         & 60.2\\
\midrule
&\emph{Additional stage of finetuning} \\
\small{8} & All + finetuning on BERT forgettables   & 70.8                     & 81.8                         & 73.5  \\
\small{9} & All + finetuning on BiLSTM forgettables &  \underline{74.0}                     & 82.5             & \underline{76.1} \\
\small{10} & \hspace{0.1cm} All + finetuning on Random $_{\balancedlstm}$       & 60.9                     & 84.4                         & 66.7  \\
\small{11} & All + finetuning on BoW forgettables    & \underline{73.7}                     & 82.4             & \underline{75.8} \\
\midrule
&\emph{From~\cite{clark2019dont}} & & & & \\
\small{12} & All & 62.4 & 84.2 & 67.8 \\
\small{13} & All (reweight) & 69.2 & 83.5 & 72.7 \\
\small{14} & Learned Mixin & 64.0 & 84.3 & 69.0\\
\small{15} & Product of Experts  & 66.5 & 84.0 & 70.8     \\
\bottomrule
\end{tabular}
\end{table*}

\iffalse
\begin{table*}
\caption{Stress test results}
\small
\label{tab:stress}
\centering
\begin{tabular}{ll llll | llll}
& Train examples & \multicolumn{4}{c}{Negation} & \multicolumn{4}{c}{Overlap}\\
& & E & C & N & Acc & E & C & N & ACC\\
\hline
\small{1} & All & 4.2 & 76.7 & 53.9 & 52.3 
& 17.0 & 77.0 & 55.6 & 55.3\\

\midrule
&\emph{Additional stage of finetuning} \\
\small{2} & All + finetuning on BiLSTM forgettables 
& 15.9 & 76.4 & 54.7 & 54.1 
& 23.6 & 76.8 & 55.6 & 56.2\\
\hline [0.76047002 0.09225789 0.53914917]
% &\emph{From~\cite{x}} & & & & \\
% \small{10} & All & 2.4 & 81.1 & 56.5 & -
% & 19.2 & 83.3 & 59.4 \\
% \small{11} & DRiFt-HYPO & 7.3 & 80.7 & 55.6 & - & 27.5 & 81.1 & 59.1 \\
% \small{12} & DRiFt-CBOW & 17.9 & 81.7 & 55.5 & - & 18.3 & 80.0 & 56.6 \\
% \small{13} & DRiFt-HAND & 4.3 & 80.6 & 55.5 & - & 15.0 & 81.9 & 57.4

\end{tabular}
\end{table*}
\fi

Our main results are presented in Table~\ref{tab:twoclass}. Lines 2 to 5 report results of fine-tuning BERT on different subsets of the MNLI dataset. This setting aligns with the setting presented in~\cite{toneva2018empirical} where the authors show that, in multiple image classification tasks, the same generalization performance can be obtained by training a model initialized randomly on its own forgettable examples. Our results suggest that this behavior may be task and/or architecture dependent: in our setting, training only forgettable examples particularly affects generalization performance on MNLI. The most extreme drop in performance is observed when BERT is only fine-tuned on its own forgettable examples (line 2) achieving an accuracy of 38.9\%. Training on BiLSTM (line 4) or BoW forgettables (line 6) examples causes a lesser drop in accuracy on MNLI although still noticeable. One of the possible reasons of the dramatic performance loss observed in line 2 is that BERT forgettables are significantly less than the counterparts from weaker baselines. 
In order to rule out this hypothesis, we train on a random subset of examples of the same size (\balancedbert, line 3). These results suggests that there is an intrinsic difficulty in BERT forgettables that deserves to be investigated in the future.
This is also the case for BiLSTM and BoW forgettables (lines 4 and 6) to some extend comparing to random samples with same size (lines 5 and 7).

In the second block of results (lines 8-11) we adopt a different strategy: we first fine-tune BERT on all the MNLI examples in order to get a reasonable prior for the task. We perform then an additional stage of fine-tuning on the subset of forgettable examples from each of the considered models. The results confirm that slightly biasing the model towards hard examples improves robustness at a slight cost in MNLI accuracy. 
Our best model is obtained by using the BiLSTM forgettable examples (line 9) achieving an accuracy of 74.0 on HANS, 
which constitute a +15.7 absolute improvement with respect to the base model in line 1 and +4.8 and +7.5 with respect to the concurrent models of \newcite{clark2019dont} and \newcite{mahabadi2019simple}. 
Results on line 10 confirm that the forgettable subsets of examples identified by BiLSTM is responsible for the improvement. 
Fine-tuning on BoW forgettables (line 11) is also comparable to BiLSTM forgettables (line 9). This shows that the choice of the weak model 
does not seem to matter and therefore does not need prior knowledge about the dataset biases.
An additional interesting observation is that BERT forgettables provide less improvement in robustness than BiLSTM or BoW. 
This seems to align with our hypothesis that weaker baselines capture more simple explanations in the dataset.







\iffalse
\section{Analysis}
\subsection{Properties of the forgettables}
Based on the results, we have:

\begin{itemize}
    \item A subset of training examples for a specific model 
     in a specific training setting
     \item The model either learns and then forget them during training at least once, or never leans them.
     \item If we start from a (particaially) randomly initialized model and train it only on the forgettables, the model does not generalize; in fact it performs worse than chance on the test examples.
    \item If we start from a pretrained model for the task, and finetune it on
    the forgettables, the model does better on HANS and worse on original test set.
    \item It is better if we pick the set of forgettables from a model with weaker architecture.
    \item Only around 50\% of BERT forgettables are forgettable by BiLSTM -- again showing that forgettables are model dependent. 
    
\end{itemize}

Some other hypothesis:
\begin{itemize}
    \item X is in F if there is a training example with different label but similar input feature. This is in fact
    similarly true for HANS examples: there are non-entailment examples with high word overlap, while there are many 
    examples in training set of MNLI with close input feature but entailment as label.
    \item the easy rules in U do not generalize to F or F is a set of outliers -- model does not learn a rule from them. 
    This is hypothesized by the poor results of models trained only on their Fs. However, eliminating these outliers hurt the model
    performance. 
    
\end{itemize}

\iffalse
\subsection{Nearest neighbors}
We take the embedding of CLS in BERT as the representation of an example. 
In BERT models, this embedding is fed to a linear classifier and optimized by cross entropy loss. 
Here, we find and look into the k-NNs of training examples for understanding what model is actually learning. 
\fi


\subsection{Remaining questions}
\begin{enumerate}
    \item How do our experiments align with the main findings of \newcite{toneva2018empirical} about
    forgettable and unforgettable examples in some image classifcation datasets:
    \begin{quote}
        a) there exist a large number of unforgettable examples, i.e.,
examples that are never forgotten once learnt, those examples are stable across seeds and strongly
correlated from one neural architecture to another; b) examples with noisy labels are among the most
forgotten examples, along with images with “uncommon” features, visually complicated to classify;
c) training a neural network on a dataset where a very large fraction of the least forgotten examples
have been removed still results in extremely competitive performance on the test set
    \end{quote}
\end{enumerate}
\fi


\section{Discussion and Conclusion}

In this paper, we introduced a novel approach based on example forgetting to build more robust models for a natural language inference task. We  finetuned a pre-trained model on a set of ``hard'' examples selected by measuring ``example forgetting''~\citep{toneva2018empirical}. We evaluated the robustness of our approach by training exclusively using the MNLI dataset and the evaluating the model on the out-of-distribution test set of HANS~\citep{linzen2019right}. We improve \bertbase and \bertlarge performance on the challenging HANS test set by more than 15\% and 5\%, respectively. Although this paper focused on natural language inference, the method is widely applicable in other tasks and contexts. This constitutes one possible direction for future work. Moreover, we plan to analyze the forgettable examples more thoroughly to understand their special properties. Finally, we will study smoother re-weighting of the training examples and re-interpret the studied approach more formally in the context of importance weighted empirical risk minimization~\citep{sugiyama2007covariate}.

% \item investigate differences between hard examples and most forgotten examples 
%     \item smoothly integrate the finetuning phase by softly reweighting examples
%     \item compute more explicit measures of correlation between forgettables and the biases
%     \item understand why BERT forgettables suck.
%     \item can we even learn to avoid biases if no example contain counter examples of those biases ? put this in discussion
% \end{itemize}


\bibliography{bibliography}
\bibliographystyle{acl_natbib}

\appendix
\section{Detailed results on HANS}
\label{sec:detailedresults}
% \newcommand\ent{$E$}
% \newcommand\neu{$N$}
% \newcommand\con{$C$}
% \newcommand{\nent}{$\neg E$}
% \newcommand{\bt}[2]{#2}

\begin{table*}[t]
\begin{tabular}{p{\textwidth}}
\toprule
\textbf{Source HANS example} \vspace{1mm} \\
\hspace{2mm} \emph{The banker thanked the tourist.} $\longarrownot\longrightarrow$ \emph{The tourist thanked the banker.} \\
\\
\textbf{Nearest neighbors by BERT} \vspace{1mm} \\
\hspace{2mm} \emph{The model simulates the size of the pool of exchangeable base cations in the soil.} \\ \hspace{2mm} $\longrightarrow$ \emph{The model simulates the size of the pool of base cations in the dirt.} \vspace{2mm} \\
\hspace{2mm} \emph{(Or click to read my summary of Wolfe's and Rose's positions.)} \\
\hspace{2mm} $\longrightarrow$ \emph{To read my summary of Wolfe's position, click.} \\
\\
\textbf{Nearest neighbors by our Robust BERT} \vspace{1mm} \\
\hspace{2mm} \emph{He sat patiently as she talked}. $\longarrownot\longrightarrow$
\emph{She sat patiently as he spoke} \vspace{2mm} \\
\hspace{2mm} \emph{And Gates and Appiah would have to be thanked for opening the door.} \\ \hspace{2mm} $\longarrownot\longrightarrow$
\emph{Gates and Appiah were thanked for opening the door.} \\
\bottomrule
\end{tabular}
\caption{Two nearest neighbors for one HANS non-entailment ($\longarrownot\longrightarrow$)  example show-casing how our robust model (line 5 in Table 2) pushes supporting $\longarrownot\longrightarrow$ training data closer compared to standard BERT (MNLI). 
To compute nearest neighbors from the BERT models, the embedding  of the special token (CLS) is assumed as the representation of an example and cosine is used as the similarity metric.}
\label{tab:NNs}
\end{table*}

\iffalse
\begin{table*}[]
\begin{tabular}{p{0.45\textwidth} p{0.45\textwidth} }
\\
\multicolumn{2}{l}{\emph{The HANS example}}
\\
\toprule
\multicolumn{2}{l}{
\begin{tabular}[c]{@{}l@{}}
\emph{The banker thanked the tourist.} $\longarrownot\longrightarrow$ \emph{The tourist thanked the banker.} \\
\end{tabular}
}
\\
\toprule
\multicolumn{1}{c}{\textbf{Nearest neighbors of BERT (MNLI)}}
& \multicolumn{1}{c}{\textbf{Nearest neighbors of our Robust BERT}}     \\
\midrule
\begin{tabular}[t]{@{}p{0.45\textwidth}@{}}
\emph{The model simulates the size of the pool of exchangeable base cations in the soil.} $\longrightarrow$ \emph{The model simulates the size of the pool of base cations in the dirt.} \\
% \\ \emph{Label}: \ent
\end{tabular} & 
\begin{tabular}[t]{@{}p{0.45\textwidth}@{}}
\emph{He sat patiently as she talked}. $\longarrownot\longrightarrow$
\emph{She sat patiently as he spoke}
\end{tabular}
\
\\
% \cmidrule(lr){2-2}
\midrule
\begin{tabular}[c]{@{}p{0.45\textwidth}@{}}
P:	(Or click to read my summary of Wolfe's and Rose's positions.)
H:	To read my summary of Wolfe's position, click.
label: \ent
\end{tabular}
&
\begin{tabular}[c]{@{}p{0.45\textwidth}@{}}
\emph{And Gates and Appiah would have to be thanked for opening the door.} $\longarrownot\longrightarrow$
\emph{Gates and Appiah were thanked for opening the door.} \\
\end{tabular}
\\
\end{tabular}
\caption{Two nearest neighbors for one HANS non-entailment (\nent{})  example show-casing how our robust model (line 5 in Table 2) pushes supporting \nent{} training data closer compared to standard BERT (MNLI). 
To compute nearest neighbors from the BERT models, the embedding 
of the special token (CLS) is assumed as the representation of
an example and cosine is used as the similarity metric.}
\label{tab:NNs}
\end{table*}
\fi


\end{document}

%TODO:
%- https://dawn.cs.stanford.edu/2019/03/22/glue/ (Signal 4: Dataset Slicing) 

% https://microsoft-my.sharepoint.com/:x:/r/personal/yayaghoo_microsoft_com/_layouts/15/Doc.aspx?sourcedoc=%7B41975454-abff-488b-8f55-c550ad840d9f%7D&action=edit&activeCell=%27compare_baseModels%27!D9&wdInitialSession=c219a303-a524-465a-bdd6-28ac21d1312c&wdRldC=1

% \begin{table}[h]
%     \centering
%     \begin{tabular}{l | l l}
%         Train source & HANS & MNLI \\
%         \hline
%         All & 60.0 & \\
%         BERT forgettables & & \\
%         BiLSTM forgettables                       & 54.0 & \\
%         BiLSTM forgettables + 20\% unforgettables & 57.5 & \\
%         BiLSTM forgettables + 30\% unforgettables & 57.5 & \\
%         BiLSTM forgettables + 50\% unforgettables & 57.5 & \\
%         BiLSTM forgettables + 80\% unforgettables & 57.5 & \\
%         \hline 
%         pretrian all, fine-tune on BERT forgettables & 67.8 & \\
%         pretrian all, fine-tune on BiLSTM forgettables & 69.2 & \\
%         pretrian all, fine-tune on BiLSTM 20\% unforgettables & 61.7 & \\
        
%     \end{tabular}
%     \caption{Caption}
%     \label{tab:my_label}
% \end{table}

